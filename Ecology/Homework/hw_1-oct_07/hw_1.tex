\documentclass{article} % A4 paper and 11pt font size
\setcounter{secnumdepth}{0}

\usepackage{amssymb, amsmath, amsfonts}
\usepackage{moreverb}
\usepackage{graphicx}
\usepackage{enumerate}
\usepackage{graphics}
\usepackage[margin=1in]{geometry}
\usepackage{color}
\usepackage{tocloft}
\renewcommand{\cftsecleader}{\cftdotfill{\cftdotsep}}
\usepackage{array}
\usepackage{float}
\usepackage{csquotes}
\usepackage{verbatim}
\usepackage{hyperref}
\usepackage{textcomp}
\usepackage[makeroom]{cancel}
\usepackage{bbold}
\usepackage{scrextend}
\usepackage{alltt}
\usepackage{listings}
\usepackage{physics}
\usepackage{mathtools}
\usepackage[normalem]{ulem}
\usepackage{amsthm}
\usepackage{tikz}
\usetikzlibrary{positioning}
\usetikzlibrary{arrows}
\usepackage{pgfplots}
\usepackage{bigints}
\allowdisplaybreaks
\pgfplotsset{compat=1.12}

\theoremstyle{plain}
\newtheorem*{theorem*}{Theorem}
\newtheorem{theorem}{Theorem}
\newtheorem*{lemma*}{Lemma}
\newtheorem{lemma}{Lemma}

\definecolor{verbgray}{gray}{0.9}
% \definecolor{dkgreen}{green}{0.9}

\lstnewenvironment{code}{%
  \lstset{
  language=R,
  backgroundcolor=\color{verbgray},
  keywordstyle=\color{blue},      % keyword style
  commentstyle=\color{magenta},   % comment style
  stringstyle=\color{olive},      % string literal styleframe=single,
  numberstyle=\color{black},      % string literal styleframe=single,
  framerule=0pt,
  numbers=left,
  stepnumber=1,
  firstnumber=1,
  showspaces=false,
  basicstyle=\ttfamily}}{}

\lstnewenvironment{console_output}{%
  \lstset{
  framerule=0pt,
  numbers=left,
  stepnumber=1,
  showspaces=false,
  firstnumber=1,
  basicstyle=\ttfamily}}{}


\makeatletter
\newcommand{\BIGG}{\bBigg@{3}}
\newcommand{\vast}{\bBigg@{4}}
\newcommand{\Vast}{\bBigg@{5}}
\makeatother

\newenvironment{definition}[1][Definition]{\begin{trivlist}
\item[\hskip \labelsep {\bfseries #1}]}{\end{trivlist}}

\newcommand{\dy}{\partial_y}
\newcommand{\dyy}{\partial_{yy}}
\newcommand{\dxx}{\partial_{xx}}
\newcommand{\dxy}{\partial_{xy}}
\newcommand{\dyyy}{\partial_{yyy}}
\newcommand{\dxxx}{\partial_{xxx}}
\newcommand{\dx}{\partial_x}
\newcommand{\E}{\varepsilon}
\def\Rl{\mathbb{R}}
\def\Cx{\mathbb{C}}

\newcommand{\Ei}{\text{Ei}}

\usepackage[T1]{fontenc} % Use 8-bit encoding that has 256 glyphs
\usepackage{fourier} % Use the Adobe Utopia font for the document - comment this line to return to the LaTeX default
\usepackage[english]{babel} % English language/hyphenation

\usepackage{sectsty} % Allows customizing section commands
\allsectionsfont{\centering \normalfont\scshape} % Make all sections centered, the default font and small caps

\usepackage{fancyhdr} % Custom headers and footers
\pagestyle{fancy} % Makes all pages in the document conform to the custom headers and footers
\fancyhead[L]{\bf Sam Fleischer}
\fancyhead[C]{\bf UC Davis \\ Principles of Population Biology (PBG200A)} % No page header - if you want one, create it in the same way as the footers below
\fancyhead[R]{\bf Fall 2016}

\fancyfoot[L]{\bf } % Empty left footer
\fancyfoot[C]{\bf \thepage} % Empty center footer
\fancyfoot[R]{\bf } % Page numbering for right footer
\renewcommand{\headrulewidth}{0pt} % Remove header underlines
\renewcommand{\footrulewidth}{0pt} % Remove footer underlines
\setlength{\headheight}{25pt} % Customize the height of the header

\newcommand{\VEC}[2]{\left\langle #1, #2 \right\rangle}
\newcommand{\ran}{\text{\rm ran }}
\newcommand{\Hilb}{\mathcal{H}}
\newcommand{\lap}{\Delta}

\newcommand{\littleo}[1]{\text{\scriptsize$\mathcal{O}$}\qty(#1)}

\DeclareMathOperator*{\esssup}{\text{ess~sup}}

\newcommand{\problem}[2]{
\vspace{.375cm}
\boxed{\begin{minipage}{\textwidth}
    \section{\bf #1}
    #2
\end{minipage}}
}

\numberwithin{equation}{section} % Number equations within sections (i.e. 1.1, 1.2, 2.1, 2.2 instead of 1, 2, 3, 4)
\numberwithin{figure}{section} % Number figures within sections (i.e. 1.1, 1.2, 2.1, 2.2 instead of 1, 2, 3, 4)
\numberwithin{table}{section} % Number tables within sections (i.e. 1.1, 1.2, 2.1, 2.2 instead of 1, 2, 3, 4)

\setlength\parindent{0pt} % Removes all indentation from paragraphs - comment this line for an assignment with lots of text

\newcommand{\horrule}[1]{\rule{\linewidth}{#1}} % Create horizontal rule command with 1 argument of height

\title{ 
\normalfont \normalsize 
\textsc{UC Davis, Principles of Population Biology (PBG 200A), Fall 2016} \\ [25pt] % Your university, school and/or department name(s)
\horrule{2pt} \\[0.4cm] % Thin top horizontal rule
\Huge Homework \#8 \\ % The assignment title
\horrule{2pt} \\[0.5cm] % Thick bottom horizontal rule
}

\author{\huge Sam Fleischer} % Your name

\date{October 7, 2016} % Today's date or a custom date

\begin{document}\thispagestyle{empty}

\maketitle % Print the title

\makeatletter
\@starttoc{toc}
\makeatother

\pagebreak

%%%%%%%%%%%%%%%%%%%%%%%%%%%%%%%%%%%%%%
\problem{Problem 1}{The whooping crane has been the target of protection efforts by the National Audubon Society and the federal governments of the US and Canada since 1941. There is one wild breeding population that winters at the Arkansas National Wildlife Refuge on the coast of Texas. The size of this population from 1941 to 1989 was studied by Dennis et al. (1991) in their Ecological Monographs paper Estimation of growth and extinction parameters for endangered species. The data set used by Dennis et al. is in the cranes.csv data file in the resources folder at SmartSite.
\begin{enumerate}[\ \ (a)]
    \item Assuming the log population numbers can be modeled by
    \begin{align*}
        \log N(t) = \log N(0) + rt + \text{``observation error''}
    \end{align*}
    where the ``observation error'' is normally distributed, use linear regression to estimate $\log N(0)$ and $r$.  Use this model to estimate $N(2011)$.
    \item Assuming the log population numbers can be modeled by
    \begin{align*}
        \log \frac{N(t+1)}{N(t)} = r + \text{``process error''}
    \end{align*}
    where the ``process error'' is normally distributed, use linear regression to estimate $r$.  Use this model to estimate $N(2011)$.
    \item Do an online search to find an estimate for the \emph{wild} whooping population size in August 2011.  How do your model predictions compare?
\end{enumerate}}
\begin{enumerate}[\ \ (a)]
    \item
        The following code snippet was used to load the CSV file and run a linear regression on the log population numbers.
        \begin{code}
# Load the file
cranes = read.csv("cranes.csv")

# Set local variables to access the data
Count = cranes$Number
Year = cranes$Year

# Plot Count vs. Year
plot(Year, Count, pch=4)

# Take the logarithm of the population numbers.
logPop = log(Count)

# Perform a linear regression
regression = lm(logPop~Y)

# Add the exponential (line in log-space) to the plot
lines(Year,exp(regression$fitted.values))

# Print the summary of the regression
summary(regression)
        \end{code}
        The output from the summary command shows $\log N(0) \approx -67.695$ and $r \approx 0.036390$, so $\log N(t) \approx -67.695 + 0.036390t$, which then implies
        \begin{align*}
            N(t) \approx N_0\exp[0.036390t] \qquad \text{with} \qquad N_0 = \exp[-67.695].
        \end{align*}
        Below is the plot generated from the code snippet.
        \begin{figure}[ht!]
            \centering
            \includegraphics[scale=0.3]{1a.png}
        \end{figure}
        Using this linear regression, $N(2011) \approx 241.12$.

    \item
        The following code snippet was used to load the CSV file and run a linear regression on the ratio of successive log population numbers.
        \begin{code}
# Load the file
cranes = read.csv("cranes.csv")

# Use the variable ``N'' as the population
N = cranes$Count

# Take the log of the population
logN = log(N)

# Get ``future'' and ``past'' arrays
logNtp1 = logN[-1]
logNt = logN[-length(logN)]

# Plot them against each other
plot(logNt,logNtp1)

# Linearly regress them and plot the result
regression = lm(logNtp1~logNt)
lines(logNt,regression$fitted.values)

# Print the result of the regression
summary(regression)
        \end{code}
        The output from the summary command shows $\log N_{t+1} \approx 0.99783\log N_t + 0.04810$, which implies
        \begin{align*}
            N_{t+1} = e^{0.04810}N_t^{0.99783}
        \end{align*}
        Using $N_{1989}$ from the data, ($N_{1989} = 146$), we get the approximation $N_{2011} = 325.53$.  Using the initial condition $N_{1938} = 19$ and the above recursive rule, we get the approximation $N_{2011} = 318.37$.

    \item
        The website {\color{blue}\hyperlink{https://www.learner.org/jnorth/tm/crane/PopulTotals.html}{https://www.learner.org/jnorth/tm/crane/PopulTotals.html}} gives the wild Whooping Crane Population in the Western Flock in 2011 as $278$.  It looks like the linear regression from (a) underestimated the most (relative error $= -13.3\%$), while the results from (b) are $325.53$ and $318.37$ (average of $321.95$), which is an overapproximation of the actual data (relative error $= 15.8\%$).
\end{enumerate}













%%%%%%%%%%%%%%%%%%%%%%%%%%%%%%%%%%%%%%
\problem{Problem 2}{\textbf{Monad's nightmare:} Starting with a single cell of \emph{E. coli}, how long would it take to cover the Earth 1 foot deep with \emph{E. coli}?  Assume that \emph{E. coli} are rectangular prisms of volume $1.1250\times10^{-18} \text{meters}^3$ and that they reproduce every 20 minutes.}

The radius of the Earth is approximately $20903520 \text{ feet}$, and thus the volume of the $1 \text{ foot}$ film of \emph{E. coli} around the Earth is
\begin{align*}
    V_{\text{film}} = \frac{4}{3}\pi 20903521^3 - \frac{4}{3}\pi 2090352103^3 = 5490965741568000 \text{ feet}^3 \approx 155486834545908.5625 \text{ meters}^3
\end{align*}
This means it would take approximately $\frac{155486834545908.5625}{1.1250\times10^{-18}} \approx 1.3821051959636316\times 10^{32}$ \emph{E. coli} cells to cover the earth with a $1$ foot film.  If we discretize time into twenty minute intervals and assume that the \emph{E. coli} cells never die, then we just need to solve $1.3821051959636316\times 10^{32} = 2^t$ for $t$, which has the solution
\begin{align*}
    t = 106.7685664639302
\end{align*}
Since this is the number of twenty minute intervals, this corresponds to about 35.6 hours.












% %%%%%%%%%%%%%%%%%%%%%%%%%%%%%%%%%%%%%%
% \problem{Problem 3}{For a closed, homogeneous population with a density-independent fitness $R$, we get $$N_{t+1} = RN_t$$
% \begin{enumerate}[\ \ (a)]
%     \item In \emph{The Origin of Species}, Darwin states \begin{displayquote}The elephant is reckoned to be the slowest breeder of all known animals, and I have taken some pains to estimate its probable minimum rate of natural increase.\end{displayquote}and continues to argue that a single breeding pair of elephants (i.e.~one female) would produce $15$ million elephants after $5$ centuries.  Assuming that the sex-ratio of elephants is $50-50$ and a time step is one year, estimate $R$.
%     \item In 1869, a correspondent of Darwin criticized Darwin's computations.  Darwin respondd \begin{displayquote}I am much obliged to your Correspondence of June 5 for having pointed out a great error in my `Origin of Species,' on the possible rate of increase of the elephant. I inquired from the late Dr. Falconer with respect to the age of breeding, etc., and understated the data obtained from him, with the intention, vain as it has proved, of not exaggerating the result. Finding that the calculation was difficult, I applied to a good arithmetician; but he did not know any formula by which a result could easily be obtained; and he now informs me that I then applied to some Cambridge mathematician. Who this was I cannot remember, and therefore cannot find out how the error arose. From the many familiar instances of rapid geometrical increase, I confess that, if the answer had been thirty or sixty million elephants, I should not have felt much surprise; but I ought not to have relied so implicitly on my mathematical friend.\end{displayquote}Darwin proceeded to give an argument that would lead to an estimate of $R \approx 1.02175$.  Using this estimate for $R$, determine how many elephants would be produced by a single breeding pair after $750$ years.
% \end{enumerate}}












%%%%%%%%%%%%%%%%%%%%%%%%%%%%%%%%%%%%%%
\problem{Problem 4}{In this problem, you will play around with the Bellows model using some of his data.  To get things started you need to download the data file {\color{red}bellows1d.txt} and the bifurcation file {\color{red}bellowsBif.R}.
\begin{enumerate}[\ \ (a)]
    \item For the data file {\color{red}bellows1d.txt}, you are going to estimate the parameters $a$, $b$, and $c$ for the survivorship function $$S(N) = \frac{aN}{1 + (bN)^c}$$ using nonlinear regression.  To this end, load the data and rename it as follows
    \begin{alltt}
        >temp=read.table("bellows1d.txt",header=FALSE) \\
        >N=temp[,1] \\
        >S=temp[,2]
    \end{alltt}
    To fit the model $S(N)$ to the data, use the nonlinear regression command
    \verbatiminput{listing1.txt}
    where the second argument provides initial guesses for the parameter values.  To see the parameter estimates, type
    \begin{alltt}
        >summary(reg)
    \end{alltt}
    To see what the fit looks like, type
    \begin{alltt}
        >plot(N,S,col="red") \\
        >lines(N,fitted.values(reg),col="blue")
    \end{alltt}
    \item With your fitted parameters in hand, simulate the Bellows model $$N_{t+1} = FS(N_t)$$ with $N_0 = 1$ for the following fecundity values: $F = 1,3,4,6,7,10$.  Discuss what happens for each of these parameter values.
    \item Determine analytically the minimum value $F$ for which the population persists.
    \item To see how the behaviors change as you vary $F$, open the R-file {\color{red}bellowsBif.R} and change the values of $a$, $b$, and $c$ in this file to what you hound in (a) and for the range of $F$ values used in (b).  Run the file.  R will produce a \emph{bifurcation diagram} that for each parameter value plots the abundances observed in the long-term in the vertical direction e.g.~if there is only one point above a given $F$ value then the population tends toward that equilibrium value; if there are two values plotted either the population oscillates between the two values or there are two stable equilibria.  Using this plot, do the following:
    \begin{enumerate}[\ \ (i)]
        \item Estimate for what range of fecundities the population goes to a positive equilibrium.
        \item Estimate for what range of fecundities the population goes to a period two orbit.
        \item Estimate for what range of fecundities the population goes to a period four orbit.
        \item Discuss whether fecundity is always stabilizing or destabilizing.
    \end{enumerate}
\end{enumerate}}

\begin{enumerate}[\ \ (a)]
    \item
        Running the given code gives the following parameter values:
        \begin{align*}
            a \approx 0.8068723 \qquad b \approx 0.0148920 \qquad c \approx 4.2470522
        \end{align*}
        so the model is
        \begin{align*}
            S(N) \approx \frac{0.8068723N}{1 + (0.0148920N)^{4.2470522}}
        \end{align*}
        and here is the plot with the data points in red dots and the model approximation as a blue line:
        \begin{figure}[ht!]
            \centering
            \includegraphics[scale=0.3]{4a.png}
        \end{figure}
    \item
        Here is the result of the model for the six different values of $F$:
        \begin{figure}[ht!]
            \centering
            \includegraphics[scale=0.3]{4b.png}
        \end{figure}

        We see that different fecundity values produce chaotic, cyclic, and equilibrium behavior, and in particular, low fecundity results in extinction.
    \item
        For the model
        \begin{align*}
            N_{t+1} = R(N_t) \qquad \text{where} \qquad R(N) = F\frac{aN}{1 + (bN)^c},
        \end{align*}
        we know $0$ is an equiilbrium since $R(0) = 0$.  To find the stability of this equilibrium, notice
        \begin{align*}
            R'(0) = Fa
        \end{align*}
        and thus $0$ is stable if $\abs{Fa} < 1$, i.e.~$F < \frac{1}{a}$ and unstable if $\abs{Fa} > 1$, i.e.~$F > \frac{1}{a}$.  Since $a \approx 0.8068723$, this critical value of $F$ is
        \begin{align*}
            F^* = \frac{1}{a} \approx 1.23935349.
        \end{align*}
    \item
        Here is the resulting plot from \texttt{BellowsBif.R}:
        \begin{figure}[ht!]
            \centering
            \includegraphics[scale=0.3]{4c.png}
        \end{figure}

        \begin{enumerate}[\ \ \ (i)]
            \item We see the population reaches equilibrium for $F \lesssim 2.35$, and specifically a positive equilibrium for $F \gtrsim 1.25$.
            \item We see period two cycles for $2.35 \lesssim F \lesssim 3.19$.
            \item We see period four cycles for $3.19 \lesssim F \lesssim 3.55$.
            \item While there is a general trend of de-stabilization as fecundity increases, the bifurcation diagram clearly shows a bifurcation at $F \approx 6.22$ where the dynamics move from chaos to period 3 cycles.  As $F$ increases, those cycles then bifurcate to period 6, then period 12, and so on, back to chaos, but there are certain intervals of $F$ which produce finite cycles bounded by regions which produce chaos.  There are many of these regions, as we can see from the bifurcation plot for $1 \leq F \leq 30$:
            \begin{figure}[ht!]
                \centering
                \includegraphics[scale=0.3]{4div.png}
            \end{figure}
        \end{enumerate}
        
\end{enumerate}












%%%%%%%%%%%%%%%%%%%%%%%%%%%%%%%%%%%%%%
\problem{Problem 5}{You can find five more population data sets in the resources folder at Smartsite in the files {\color{red}data1.txt}, {\color{red}data2.txt}, {\color{red}data3.txt}, {\color{red}data4.txt}, {\color{red}data5.txt}.  For each of these data sets use Bayesian Information Criterion (BIC) to select one of the following three models where the ``process error'' is assumed to be normally distributed:
\begin{addmargin}[1em]{2em}
    \textbf{Gompertz model:} is of the form $$\log\frac{N(t+1)}{N(t)} = a + b\log N(t) + \text{``process error''}$$

    \textbf{Ricker model:} is of the form $$\log\frac{N(t+1)}{N(t)} = a + b N(t) + \text{``process error''}$$

    \textbf{Random walk model:} is of the form $$\log\frac{N(t+1)}{N(t)} = a + \text{``process error''}$$
\end{addmargin}}

Here is the code used to generate the parameter values for each model, as well as the Log-Likelihood, AIC, and BIC values:

\begin{code}
# Define the Ricker, Gompertz, and Random Walk models
RICKER = function(x, a, b) x + a + b*exp(x)
GOMPERTZ = function(x, a, b) a + (b+1)*x
RANDOM_WALK = function(x, a) x + a

# Define a list of files to load
data_list <- c("data1.txt", "data2.txt", "data3.txt", "data4.txt", "data5.txt")

for (FILE in data_list) {
  # Load the file
  data = read.csv(FILE)
  N = data$N

  # Take the logarithm of the data
  y = log(N)
  k = length(y)
  y.future = y[-1]
  y.past = y[-k]
  
  # Perform the regressions
  MODEL_GOMPERTZ = nls(y.future~GOMPERTZ(y.past, a, b), start=list(a=0, b=0))
  MODEL_RICKER = nls(y.future~RICKER(y.past, a, b), start=list(a=0, b=0))
  MODEL_RANDOM_WALK = nls(y.future~RANDOM_WALK(y.past, a), start=list(a=0))
  
  # Print the name of the file
  print(FILE)
  # Print the Coefficients
  print("COEFFICIENTS")
  print(coef(MODEL_GOMPERTZ))
  print(coef(MODEL_RICKER))
  print(coef(MODEL_RANDOM_WALK))
  # Print the Log-Likelihood values
  print("LOGLIK")
  print(logLik(MODEL_GOMPERTZ))
  print(logLik(MODEL_RICKER))
  print(logLik(MODEL_RANDOM_WALK))
  # Print the AIC and BIC values
  print("AIC")
  print(AIC(MODEL_GOMPERTZ))
  print(AIC(MODEL_RICKER))
  print(AIC(MODEL_RANDOM_WALK))
  print("BIC")
  print(BIC(MODEL_GOMPERTZ))
  print(BIC(MODEL_RICKER))
  print(BIC(MODEL_RANDOM_WALK))
}
\end{code}

Here are the results from the various regressions:

\begin{table}[ht!]
    \begin{tabular}{||l||l|l|l|l|l||}\hline\hline
        data1.txt   & $a$ & $b$ & Log-Likelihood & AIC & BIC \\\hline\hline
        Gompertz    & $2.549111$ & $-0.538343$ & $-68.97711$ & $143.9542$ & $149.4401$ \\\hline
        Ricker      & $0.794930970$ & $-0.003908944$ & $-63.17832$ & $132.3566$ & $137.8426$ \\\hline
        Random Walk & $-0.02756187$ & N/A & $-76.10542$ & $156.2108$ & $159.8681$ \\\hline\hline
    \end{tabular}
\end{table}

\begin{table}[ht!]
    \begin{tabular}{||l||l|l|l|l|l||}\hline\hline
        data2.txt   & $a$ & $b$ & Log-Likelihood & AIC & BIC \\\hline\hline
        Gompertz    & $2.2840263$ & $-0.6554932$ & $-11.72863$ & $29.45726$ & $34.74086$ \\\hline
        Ricker      & $0.57313072$ & $-0.01663231$ & $-12.86428$ & $31.72857$ & $37.01217$ \\\hline
        Random Walk & $-0.0035849$ & N/A & $-20.00474$ & $44.00948$ & $47.53188$ \\\hline\hline
    \end{tabular}
\end{table}

\begin{table}[ht!]
    \begin{tabular}{||l||l|l|l|l|l||}\hline\hline
        data3.txt   & $a$ & $b$ & Log-Likelihood & AIC & BIC \\\hline\hline
        Gompertz    & $1.2411181$ & $-0.2311494$ & $-2.360294$ & $10.72059$ & $14.12707$ \\\hline
        Ricker      & $0.736252393$ & $-0.003664602$ & $-3.534877$ & $13.06975$ & $16.47624$ \\\hline
        Random Walk & $0.1956439$ & N/A & $-12.05806$ & $28.11611$ & $30.3871$ \\\hline\hline
    \end{tabular}
\end{table}

\begin{table}[ht!]
    \begin{tabular}{||l||l|l|l|l|l||}\hline\hline
        data4.txt   & $a$ & $b$ & Log-Likelihood & AIC & BIC \\\hline\hline
        Gompertz    & $0.6757822$ & $-0.1522380$ & $24.10618$ & $-42.21236$ & $-38.6782$ \\\hline
        Ricker      & $0.162914901$ & $-0.001894426$ & $24.00982$ & $-42.01963$ & $-38.48547$ \\\hline
        Random Walk & $0.008361279$ & N/A & $22.87781$ & $-41.75561$ & $-39.39951$ \\\hline\hline
    \end{tabular}
\end{table}

\begin{table}[ht!]
    \begin{tabular}{||l||l|l|l|l|l||}\hline\hline
        data5.txt   & $a$ & $b$ & Log-Likelihood & AIC & BIC \\\hline\hline
        Gompertz    & $1.1102957$ & $-0.3326013$ & $-113.7893$ & $233.5786$ & $241.078$ \\\hline
        Ricker      & $0.332103758$ & $-0.007314065$ & $-118.2062$ & $242.4124$ & $249.9119$ \\\hline
        Random Walk & $-0.002411484$ & N/A & $-121.9505$ & $247.901$ & $252.9006$ \\\hline\hline
    \end{tabular}
\end{table}

We choose the Ricker Model for the first data set and the Gompertz Model for the second, third, fourth, and fifth data sets since these models produce the lowest BIC scores for the data.













%%%%%%%%%%%%%%%%%%%%%%%%%%%%%%%%%%%%%%
\problem{Problem 6}{Consider a population who in the absence of predation exhibits exponential growth i.e.~$G(N) = rN$ without predation.  If this population is subject to a predator with a saturating function response, the predation would be of the form $$\frac{aNP}{H + aN}$$ where $P$ is the predator abundance (assumed to be constant), $a$ is the searching efficiency of the predator and $H$ is the predator's ``half-saturation constant''.  In the presence of predation, the population growth rate is given by $$G(N) = rN - \frac{aN}{H + aN}.$$  Let $R(N) = G(N)/N$ be the per-capita growth rate term.
\begin{enumerate}[\ \ (a)]
    \item Find the maximum per-capita growth rate of the population.  Discuss under what conditions the population is doomed to extincton for all initial conditions.
    \item Find the minimum per-capita growth rate of the population.  Discuss under what conditions the population persists for all positive initial conditions.  Is the population regulated?
    \item Assume that the maximum per-capita growth rate of the population is positive and the minimum per-capita growth rate is negative.  Find the positive equilibrium for the model.  What happens for populations whose initial abundance lies below this equilibrium?  What happens if the initial abundance lies above this equilibrium?  Discuss how the resilience of this population can be improved (by decreasing the threshold value).
\end{enumerate}}

\begin{enumerate}[\ \ (a)]
    \item
        We see $R(N)$ is a strictly increasing function of $N$, and thus the maximum per-capita growth rate occurs as $N\rightarrow\infty$, i.e. $\displaystyle\lim_{N\rightarrow \infty}R(N) = r$.  If $r < 0$, the population is doomed to extinction for all initial conditions.
    \item
        Since $R$ is an increasing function of $N$, $R$ attains its minimum on the boundary $N=0$.  This minimum is $R(0) = r - \dfrac{a}{H}$.  If $r > \dfrac{a}{H}$ then the per-capita growth rate, at its worst, is positive, and thus the population persists for all persitive initial conditions.  The per-capita growth rate at large densities is positive and bounded from zero, so no, the population is not regulated.
    \item
        Assuming $0 < r < \dfrac{a}{H}$, then $R(0)<0$ and $R(\infty)>\infty$.  The positive equilibrium is th value of $N$ for which
        \begin{align*}
            r = \frac{a}{H + aN^*} \qquad \text{which simplifies to} \qquad N^* = \frac{a - Hr}{ar}.
        \end{align*}
        Since $R(N)$ is an increasing function, $R(N) < 0$ for $N < N^*$ and $R(N) > 0$ for $N > N^*$.  This means $N^*$ is a threshold value below which the population must not fall if it intends on surviving.  If a population density is initially below this threshold, it will approach extinction asymptotically.  If a population density is initially above this threshold, it will grow without bound.  Note that $N^*$ is a decreasing function of $H$, and $N^*$ it is not monotonic in $a$ or $r$.  This means we can surely decrease the threshold value by increasing $H$.  Decreasing the threshold value allows the population to undergo larger perturbations without increasing the risk of extinction.  This means the population is more resilient to these large perturbations.
\end{enumerate}

\end{document}
