\documentclass{article}


\usepackage[margin=0.6in]{geometry}
\usepackage{amssymb, amsmath, amsfonts}
\usepackage{mathtools}
\usepackage{physics}
\usepackage{enumerate}
\usepackage{cancel}
\usepackage{array}
\newcommand{\Rl}{\mathbb{R}}
\newcommand{\prob}{\mathbb{P}}
\newcommand{\cov}{\text{cov}}
\newcommand{\vari}{\text{var}}
\newcommand{\cor}{\text{cor}}
\newcommand{\expec}{\mathbb{E}}
\newcommand{\f}[3]{#1\ :\ #2 \rightarrow #3}

\title{PBG 200A Notes}
\author{Sam Fleischer}
\date{October 17, 2016}

\begin{document}
    \maketitle

    \section{From Last Time}
        \begin{align*}
            N(t+1) = A(t+1)N(t)
        \end{align*}
        where
        \begin{align*}
            \bar{A} = \expec[A(t+1)], \qquad B(t+1) = A(t+1) - \expec[A(t+1)],
        \end{align*}
        and
        \begin{align*}
            A(t+1) = \bar{A} + B(t+1)
        \end{align*}
        so
        \begin{align*}
            r \approx \log \lambda - \frac{1}{2\lambda^2}\sum_{i,j,k,\ell}S_{ij}S_{k\ell}\cov[B_{ij}(t)B_{k\ell}(t)]
        \end{align*}
        where $S_{ij} = \dfrac{\partial \lambda}{\partial \overline{a}_{ij}}$ and $\lambda$ is the dominant eigenvalue for $\overline{A}$.

    \section{Demographic Stochasticity}
        Populations consist of a finite number of individuals whose fates are not perfectly correlated.  The simplest model is a branching process.

        \subsection{Branching Process}
            \begin{align*}
                N(t+1) = X_1 + \dots + X_{N(t)}
            \end{align*}
            where $X_i$ are i.i.d.~on $\mathbb{N}$.  If the $X_i$ are independent of $N(t)$ - this is a standard branching process.

            There is a limit theorem for branching processes: Define $R \coloneqq \expec[X_1]$.  If $R \leq 1$, then $N_0 \rightarrow 0$ in finite time with probability $1$.  Note that $R$ is defined this way since
            \begin{align*}
                \expec[N(t+1)|N(t)] = RN(t)
            \end{align*}
            If $R > 1$, then $N_t \rightarrow \infty$ with positive probability $p<1$ and goes extinct with probability $1 - p$.  The probability of extinction when $R>1$ is then $(1 - p)^{N_0}$.

        \subsection{A Useful Approximation}
            We say $N(t+1) = X_1 + \dots + X_{N(t)}$ is approximated by a normal curve with mean $RN(t)$ and variance $N(t)\vari[X_1]$ (denot $\sigma^2 \coloneqq \vari[X_1]$).
            \begin{align*}
                N(t+1) = X_1 + \dots + X_{N(t)} \approx \underbrace{RN(t) + Z\sigma\sqrt{N(t)}}_{= N(t)\qty(R + \frac{\sigma}{\sqrt{N(t)}}Z)}
            \end{align*}
            This is called the diffusion approximation.  This means smaller populations are at greater extinction risk.  What happens to population size on the event of non-extinction?
            \begin{align}
                \expec[N(t+1)] = RN(t)
            \end{align}
            but
            \begin{align}
                \expec[N(t+1)] = \expec[N(t+1)|N(t+1) > 0]\overbrace{\prob[N(t+1)>0]}^{<1} + \cancelto{0}{\expec[N(t+1)|N(t+1)=0]}\prob[N(t+1)=0]
            \end{align}



\end{document}















