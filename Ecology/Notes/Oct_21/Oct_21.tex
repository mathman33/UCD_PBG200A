\documentclass{article}


\usepackage[margin=0.6in]{geometry}
\usepackage{amssymb, amsmath, amsfonts}
\usepackage{mathtools}
\usepackage{physics}
\usepackage{enumerate}
\usepackage{cancel}
\usepackage{array}
\newcommand{\Rl}{\mathbb{R}}
\newcommand{\prob}{\mathbb{P}}
\newcommand{\cov}{\text{cov}}
\newcommand{\vari}{\text{var}}
\newcommand{\cor}{\text{cor}}
\newcommand{\expec}{\mathbb{E}}
\newcommand{\f}[3]{#1\ :\ #2 \rightarrow #3}

\title{PBG 200A Notes}
\author{Sam Fleischer}
\date{October 17, 2016}

\begin{document}
    \maketitle

    \section{From Last Time}
        \begin{itemize}
            \item Pulliam 1988 - under equilibrium conditions, you get sources and sinks, or you get a balanced patch landscape (ideal free distribution (ex. ideal pike system))
            \item Levins - many patches (occupied patches, spacial heterogeneity)
        \end{itemize}
        ``Should I Stay or Should I Go?'' - The Clash

    \section{Background - Evolution of Dispersal}
        Basic result - spacial heterogeneity selects against dispersal.  Temporal heterogeneity is neutral.  Spacial AND temporal heterogeneity allows for possibility of selection \emph{for} dispersal.

    \section{Levins' Model}
        Assumptions
        \begin{itemize}
            \item same size and quality patches
            \item dispersing randomly
            \item every patch is equally connected to every other patch
            \item infinite number of patches
        \end{itemize}

    \section{Incidence Function Models (IFMs)}
        Finite number of patches, spacially explicit.  Distance between patches $i$ and $j$ is $d_{ij}$.  Patch $i$ has area $A_i$.  If patch $i$ is occupied, it goes extinct at a rate $A_i^{-x}$.  The larger the patch, the slower it goes extinct.  Patch $i$ becomes occupied at rate $\displaystyle\sum_{j \text{ occupied}} cA_jA_i\exp[-d_{ij}a]$.  This is the net propagule pressure on the focal patch (patch $i$), and is the rate at which it becomes colonized.

        \subsection{Mean Field Model}
            $p_i$ is the probability patch $i$ is occupied at any particular point in time.
            \begin{align}
                \frac{\dd p_i}{\dd t} = -p_iA_i^{-x} + (1 - p_i)\sum_{j \text{ occupied}} c A_jA_i\exp[-d_{ij}a]
            \end{align}
            Deterministic approximation of probabilistic model.

    \section{Rescue Effect}
        Let $p$ be the fraction of occupied patches. Levins' Model is
        \begin{align}
            \frac{\dd p}{\dd t} = cp\qty(s - p) - \frac{ep}{1 + ap}
        \end{align}
        where $a$ is the strength of the rescue effect.  Positive equilibria satisfy $c(s - p) = \dfrac{e}{1 + ap}$ (colonization rate matches extinction rate).  Thus,
        \begin{align}
            \frac{c}{e} = \frac{1}{\qty(1 + ap)\qty(s - p)}
        \end{align}
        Plot bifurcation diagram ($p^*$ vs.~$\frac{c}{e}$): $0$ is always an equilibrim (plot $p^* = 0$).  Horizontal asymptotes at $s$ and $-\frac{1}{a}$.  Sideways-facing parabola(ish) thing approaching the asymptotes.  Get stable states for certain values of $\frac{c}{e}$.


\end{document}















