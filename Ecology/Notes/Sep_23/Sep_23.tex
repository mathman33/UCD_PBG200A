\documentclass{article}


\usepackage[margin=0.6in]{geometry}
\usepackage{amssymb, amsmath, amsfonts}
\usepackage{mathtools}
\usepackage{physics}
\usepackage{enumerate}
\usepackage{array}
\newcommand{\Rl}{\mathbb{R}}
\newcommand{\f}[3]{#1\ :\ #2 \rightarrow #3}

\title{PBG 200A Notes}
\author{Sam Fleischer}
\date{September 23, 2016}

\begin{document}
    \maketitle

    \section{Discrete-Time, Density-Independence}

    \begin{align*}
        N_{t+1} - N_t = (b - d)N_t \\
        \implies N_{t+1} = (1 + b - d)N_t \coloneqq RN_t
    \end{align*}
    if $N_0$, then $N_1 = RN_0$, $N_2 = RN_1 = R^2N_0$, and so $N_t = R^tN_0$.

    So from before, $e^r = R$.  We will always use the variable $r$ to denote intrinsic growth rate

    \section{Negative Density Dependence}

    This means $r$ is decreasing with density.  Examples:
    \begin{itemize}
        \item higher populations mean easier spread of pathogens
        \item predator has type 3 functional response
        \item cap of resources
        \item accumulation of waste producs
    \end{itemize}
    yeast model, T Carlson, 1913

    picking functions to ``graduate'' a set of data.. the choice is, at its very best, only a combination of good judgement and good luck.

    Simplest case of decrease function is a linear function, 
    $r(N) = r_0\qty(1 - \frac{N}{K})$, and so we obtain the logistic growth model:
    \begin{align*}
        \dot{N} = Nr(N) = r_0N\qty(1 - \frac{N}{K})
    \end{align*}
    This can be solved using separation of variables.. the solution is:
    \begin{align*}
        FILL\ \ IN\ \ LATER
    \end{align*}
    Set $G(N) = Nr(N)$.  Then we get a downward-facing parabola with $G(0) = G(K) = 0$.  The vertex is at $\frac{K}{2}$.  So $0$ and $K$ are equilibria, and there is an inflection point at $\frac{K}{2}$ on the $N$ vs. $t$ graph.  All trajectories with $N(0) > 0$ approach $K$ asymptotically.

    Fitting linear decrease of growth rate to the 1913 data gives unbelievably accurate logistic growth.

    Many populations exhibit (kind of) logistic growth.  There are fluctuations...

    \subsection{Discrete-Time Version}
    \begin{align*}
        N_{t+1} = \exp[r(N_t)]N_t
    \end{align*}
    with $r(N) \coloneqq r_0\qty(1 - \frac{N}{K})$.  This is called the ``Ricker Equation.''
    The equilbiria are still $0$ and $K$.

    (went to RStudio...)

    \subsubsection{Lyapunov Exponent}

    For Models of the form
    \begin{align}
        N_{t+1} = F(N_t),
    \end{align}
    The Lyapunov Exponent for initial condition $N_0$ is
    \begin{align}
        \chi = \lim_{T\rightarrow\infty} \frac{1}{T}\sum_{t=0}^{T-1}\log\abs{F'(N_t)}
    \end{align}

    \section{Discussion of Tilman/Wedin and Bennica et.~al.}

    \section{How to Test for Density-Dependence?}

    Suppose we have some data $y_1, y_2, \dots, y_T$, $\log$-densities.

    Which model parameter(s) $a$ are best at describing the data?
    \begin{align}
        x_{t+1} = g(x_t, a)
    \end{align}
    We assume there is some uncertainty in either data or descriptor..

    Consider two types of ``noise'':
    \begin{itemize}
        \item Observational error, i.e.~the model is perfect, but data is imperfect:
        \begin{align}
            y_t = x_t + {E_t^{\text{o}}}
        \end{align}
        where $E_t^\text{o}$ is observational error
        \item Process error, i.e.~the observations are perfect, but descriptor is imperfect:
        \begin{align}
            x_{t+1} = g(x_t,a) + E_{t+1}^\text{p}
        \end{align}
        where $E_{t+1}^\text{p}$ is process noise.
    \end{itemize}

    For Monday, read Knape and Divalpine 2012.

\end{document}