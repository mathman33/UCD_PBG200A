\documentclass{article}


\usepackage[margin=0.6in]{geometry}
\usepackage{amssymb, amsmath, amsfonts}
\usepackage{mathtools}
\usepackage{physics}
\usepackage{placeins}
\usepackage{nicefrac}
\usepackage{textcomp}
\usepackage{enumerate}
\usepackage{cancel}
\usepackage{array}
\usepackage{color}
\newcommand{\Rl}{\mathbb{R}}
\newcommand{\qiq}{\ \ \ \implies\ \ \ }
\newcommand{\vari}[1]{\text{var}\qty[#1]}
\newcommand{\cor}[1]{\text{cor}\qty[#1]}
\newcommand{\expec}[1]{\mathbb{E}\qty[#1]}
\newcommand{\cov}[1]{\text{cov}\qty[#1]}
\newcommand{\f}[3]{#1\ :\ #2 \rightarrow #3}
\newcommand{\prob}[1]{\mathbb{P}\qty[#1]}
\newcommand{\half}{\nicefrac{1}{2}}
\newcommand{\fold}[1]{$#1\times$}

\title{PBG 200A Notes}
\author{Sam Fleischer}
\date{December 5, 2016}

\begin{document}
    \maketitle

    \begin{itemize}
        \item SNP array genotyping
        \begin{itemize}
            \item based on DNA-DNA hybridization
            \item need to know where the variable sites are in an organism
            \begin{itemize}
                \item multiple prob pairs for each variable nucleotide
            \end{itemize}
            \item the individual who carries a C(A) will hybridize better to probes with G(T).  Measuring differences in signal intensities.
            \item very precise, very cost-effective, on a per-individual, per-SNP basis
            \item only is useful in model organisms
            \item For \$600, get 1 million alleles in 1 individual.
        \end{itemize}
        \item Bead array genotyping
        \begin{itemize}
            \item most common for nonmodel organisms,
            \item but have to have a reference genome and know the variable sites and alternative alleles at a site
            \item need the same info for nonmodel and model organisms, but for nonmodel organisms you need to make your own reagents.
            \item primers attach to reverse-primers which have their own barcode.  Reverse primers attach to beads, so each type of bead corresponds to a single allele.  Plate-reader fishes out the beads and immobilizes them in different locations on a plate.  Image the plate for the different tags.
            \item advantage: flexibility.
            \begin{itemize}
                \item $\$.2$ - $\$.5$ per SNP per genotype (48-384 SNPs)
                \item minimum experiment: approx 500 individuals, \$10,000-\$15,000.
                \item high requirements for DNA quantity and quality
            \end{itemize}
        \end{itemize}
        \item Sequenom MassArray genotyping
        \begin{itemize}
            \item have to know a reference sequence (something)
            \item need to know 100-200 variable sites
            \item do PCR for every locus
            \item no tags, no beads, directly reading the sequence with a mass spectrometer (must be able to determine each allele from each allele in all sites)
            \item main limitation is SNP compatibility
            \item 30-36 SNPs per plex.
            \item flexible: suitable for gradual experiments (feeds information a little bit at a time, can improve experimental design on the fly)
            \item main limitation is SNP compatibility
            \item \$.15-\$.25 per SNP per genotype
            \item requires less DNA and less quality
        \end{itemize}
        \item Genotyping by Sequence
        \begin{itemize}
            \item Do reads for $k$ individuals in a single pool
            \item separate bioinformatically later on
            \item assign ancestry to each individual at each location
        \end{itemize}
        \item Multiplexed shotgun genotyping
        \begin{itemize}
            \item essentially getting unlimited info
            \item assuming the individuals have high LD
            \item if the density of variate sites is much less than the density of recombination break points.
            \item this means we don't need high coverage - we don't have complete genome
            \item can be very cheap on the per individual basis - can multiplex multiple individuals on a single Illumina lane.
            \item very cheap - \$20-\$40 per individual fo thousands of markers
            \item disadvantages:
            \begin{itemize}
                \item need to have fairly well-assembled reference genomes, and info on each parent,
                \item high upfront costs( barcode adaotors - need a single barcide for each individual).  The same 100 or so barcodes can be reused.  Scalaing up is easy.
            \end{itemize}
            \item Key point: not using genetic information to infer a genetic map.  Using physical genome as reference.
        \end{itemize}
        \item Bulked segregant analysis
        \begin{itemize}
            \item Basic principle: identify differences in the frequencies of marker alleles between pools of individuals with different phenotypes
            \item In the absence of population structure, this is useful (do we want most or least diverse group?)
            \item can be done in lab or in natural populations
            \item limit by scale of LD
            \item Genetic designs for BSA
            \begin{itemize}
                \item controlled lab cross
                \item hitchhiking mapping (select for phenotypes, see which genotypes come with it)
                \item introgression mapping
                \item Natual poulaition
            \end{itemize}
        \end{itemize}
        \item Genotyping by sequencing (Gar paper): technology
        \begin{itemize}
            \item a lot easier and simpler - can't use it for genome mapping without a refernce equilibrium
            \item this was an excellent example of how to exploit the available technological resources
            \item pseudogenomes (updated genome) - mapping to a semi-related genome - assume the genomes are close enough, and that they are more or less colinear.  Map the reads to the semi-related genome - very high error rates, but fine for their purpose
            \item masking polymorphisms - throwing out over 90\% of data, still end up with hundreds of variable site per Mb.
        \end{itemize}
        \item Measuring gene expression: microarrays vs. RNA-seq
        \begin{itemize}
            \item what genes are expressed where, when, and in what amount
            \item determines the function of the genome
            \item look at gene expression to determine the link between genotype and phenotype
            \item Microarrays
            \begin{itemize}
                \item immobilize probes on solid support
                \item ``analog'' technology
                \item cost-effective when good arrays already available,
                \item but custom arrays can be made for any organism
                \item has almost completely been displaced by...
            \end{itemize}
            \item RNA-seq
            \begin{itemize}
                \item ``digital'' technology
                \item can be done for any organism, but requires a reference for mapping reads
                \item assuming the number of reads you observe in an RNA sample is proportional to expression
                \item don't need an organism-specifi reagent, but do need a reference genome,
                \item but can make a custom reference (which is a helluva lot easier than making a custom array)
                \item got cheaper fast, now even cheaper and easier
            \end{itemize}
        \end{itemize}
        \item Design and analysis of expression experiments
        \begin{itemize}
            \item interested in differences between treatments/categories
            \begin{itemize}
                \item brain/liver, grain/rice, male/female, etc.
                \item must account for variation within treatment first!
            \end{itemize}
            \item biological variation within treatment
            \begin{itemize}
                \item genetic differences, environmental conditions, etc.
            \end{itemize}
            \item technical variance
            \begin{itemize}
                \item effect of experimental procedures, dissection, batch of arrays/dyes,reagents, ozone levels that day, etc.
                \item Artyom knows a statistician who swears she can look at two data sets and determine which Illumina machine it came from..... wtf.
            \end{itemize}
            \item biological replicates necessary to detect differences
            \begin{itemize}
                \item avoid confounding biological and technical variation
                \item know the technical properties of your method
                \item anticipate sources of biological and technical variance
            \end{itemize}
        \end{itemize}
        \item design of RNA-seq experiments
        \begin{itemize}
            \item unlike microarrays, RNA-seq provides categorical (counts) data
            \item in principle, can detect ``differential expression'' from a single replicate per treatment (conditioned on depth)
            \item does not tell you whether you are looking at true biological variation
            \item for quantitative experiments, need multiple biological replicates per treatment, just a with arrays
            \item minimize environmental and technical variation
            \item \emph{avoid confounding biological and technical variation}
            \item example of a bad design:
            \begin{itemize}
                \item using different reagents, different Illumina machines - completely confounding biological and technical variation
            \end{itemize}
            \item example of a good design:
            \begin{itemize}
                \item use barcodes!
                \item isolate RNA - use the same batch of reagents for each library
                \item can mix all six biological replicates because of barcodes
                \item put the mix on six different lanes, minimizing technical variation
            \end{itemize}
            \item the importance of this depends on the level of variation you are measuring
            \item may be looking at subtle quantitative difference
        \end{itemize}
        \item problems with RNA-seq quantification
        \begin{itemize}
            \item standard Illumina procedure includes many enzymatic steps with purification in between
            \item losses and biases at every step
            \item OK if looking for big differences; may be a problem if looking for subtle quantative variation
        \end{itemize}
        \item Allele-specific expression analysis
        \begin{itemize}
            \item Differences in gene expression between genotpes can be due either to mutation in that gene (cis-regulatory) or any changes in its upstream regulators (trans-regulatory)
            \item For some applications, we don't care (health)
            \item In evolutionary terms, however, it makes a big difference.  How do we figure out where the divergence is based on cis- or transregulatory variation
            \item F1 hybrids can be used to infer cisregulatory variation
            \item By comparing two parents and their F1, can decompose gene expression divergence into cis- and trans0regulatory components
        \end{itemize}
    \end{itemize}

\end{document}










