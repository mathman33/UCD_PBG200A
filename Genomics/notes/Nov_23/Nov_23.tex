\documentclass{article}


\usepackage[margin=0.6in]{geometry}
\usepackage{amssymb, amsmath, amsfonts}
\usepackage{mathtools}
\usepackage{physics}
\usepackage{placeins}
\usepackage{nicefrac}
\usepackage{enumerate}
\usepackage{cancel}
\usepackage{array}
\usepackage{color}
\newcommand{\Rl}{\mathbb{R}}
\newcommand{\qiq}{\ \ \ \implies\ \ \ }
\newcommand{\vari}[1]{\text{var}\qty[#1]}
\newcommand{\cor}[1]{\text{cor}\qty[#1]}
\newcommand{\expec}[1]{\mathbb{E}\qty[#1]}
\newcommand{\cov}[1]{\text{cov}\qty[#1]}
\newcommand{\f}[3]{#1\ :\ #2 \rightarrow #3}
\newcommand{\prob}[1]{\mathbb{P}\qty[#1]}
\newcommand{\half}{\nicefrac{1}{2}}

\title{PBG 200A Notes}
\author{Sam Fleischer}
\date{November 23, 2016}

\begin{document}
    \maketitle

    \section{Genomics for Evolution and Ecology}
        Why?
        \begin{itemize}
            \item pop gen
            \item evo and eco genetics
            \item phylogeny reconstruction
            \item speciation genetics
            \item phylogeography
            \item molecular evolution
            \item comaprative genomics
            \item metagenomics
        \end{itemize}
        How?  Tons of data.  Because gathering data was difficult, a lot of thought was put into designing the perfect experiment.  Today, biology is data-rich.  Data is cheaper than brain capacity.  A lot less work goes into experimental design. \\

        It used to be really expensive.. 3 billion dollars.  Now we can sequence a human genome in 2 weeks for 10,000 dollars.  DNA is extremely dense information.  It's cheaper to keep the DNA and sequence it than it is to house the DNA in a hard drive... \\

    \section{Sequencing technologies}
        what matters?
        \begin{itemize}
            \item output capacity
            \item scalability (smallest/biggest possible experiment)
            \item read length
            \item accuracy/error profiles
            \item speed
            \item ease of prep
            \item applications.compatibility
            \item cost
            \item availability
            \item tradeoffs - choose the best for what you actually need.
        \end{itemize}

    \section{Workflow}
        \begin{itemize}
            \item get material (DNA, RNA, etc)
            \item prepare sequencing library
            \item sequence it
            \item manage/store the data - convert to raw sequence
            \item pre-process the data
            \item data reduction / assembly
            \item primary analysis (identify basic units)
            \item do science
        \end{itemize}

    \section{Types of sequencing libraries}
        \begin{itemize}
            \item DNA/RNA
            \item RNA
            \begin{itemize}
                \item normalized/unnormalized, full-length vs endtags, stranded or nonstranded
            \end{itemize}
            \item DNA
            \begin{itemize}
                \item single-end frags
                \item paired-end frags
                \item mate-pair
                \item strobed
                \item whole genome amplification?
            \end{itemize}
        \end{itemize}

    \section{Basic library prep}
        \begin{itemize}
            \item fragment the DNA - chop up into manageable pieces (acoustic shearing - ultrasonic vibrations break it up into pieces) or (enzymatic - generates double-stranded breaks)
            \item size selection (decide what you need and get it) - we want the fragment distribution to be as tight as possible - we waste a lot of material
            \item end-repair (fragments have ragged ends - must repair them), A-tail, ligate adapters (tech specific) (must be able to manipulate the fragments) very very important
            \item adapters allow capture/manipulation, ID, sequencing
            \item enrich and/or capture adapter-ligated fragments
            \item quantify
            \item load on the machine
        \end{itemize}

    \section{Basic Illumina approach}
        \begin{itemize}
            \item requires adapter ligation and 2 PCR steps
            \begin{itemize}
                \item pre sequencing (adapter enrichment)
                \item on the machine (cluster generation)
            \end{itemize}
            \item many enzymatic steps
            \item consequences: losses, duplicates, biases, errors
            \item sequencing by synthesis (multiple fluorophores)
            \item modifications: paired end, mate-pair libraries
            \item 
            \begin{itemize}
                \item start with RNA
                \item convert to double stranded cDNA
                \item add adapters on the end of the fragments (different adapters to each end)
            \end{itemize}
        \end{itemize}

    \section{Improved library prep: transposomes}
        \begin{itemize}
            \item make transposomes - get fragmented DNA with adapters on the ends
            \item advantages
            \begin{itemize}
                \item less input
                \item single tube reaction
                \item smaller volume
                \item faster prep
                \item fewer steps (fewer losses)
            \end{itemize}
        \end{itemize}

    \section{mate pair libraries}
        \begin{itemize}
            \item necessary for complex templates with many repeats
            \begin{itemize}
                \item many repeats are 10-12kb, but we sequence in 100 base pair size frags
            \end{itemize}
            \item Illumina mate-pair approach can get around long repeats
            \item Cre/lox site-specific recombination
        \end{itemize}

    \section{Illumina sequencing Bridge Amplification}
        \begin{itemize}
            \item same principle as a florescent microscop
            \begin{itemize}
                \item attach primers to fragments
                \item attach fragments to machine
                \begin{itemize}
                    \item vast majority of adapters are empty
                \end{itemize}
                \item Convert to double stranded
                \item repeat until you get clusters which can generate optical signals
            \end{itemize}
            \item Error rate in Illumina technology gets larger with the length of the fragment.
            \item Not sequencing a single molecule - rather, a cluster of 200-basepair bits
        \end{itemize}

    \section{Coverage - How much material do we need?}
        \begin{itemize}
            \item 1pcg $\approx$ 1Gb
            \item Human
            \begin{itemize}
                \item $\approx$ 6Gb (2n)
                \item 100 ng $\approx$ 20,000 cells
                \item 20 mcg - 1mg of tissue
                \item coverage?
                \begin{itemize}
                    \item HiSeq (full run) $\approx$ 100x .. 2 weeks
                    \item MiSeq (full run) $\approx$ 2x .. 2 days
                \end{itemize}
            \end{itemize}
            \item Drosophila $\approx$ 0.18Gb
            \item Some plants $>$ 100Gb
        \end{itemize}

    \section{454 Sequencing}
        \begin{itemize}
            \item First practical tech
            \item attach identical frags to a bead
            \item generate an emulsion?
            \item pipette the beads - every well is only big enough for a single bead
            \item end up with an environment where most are empty
            \item take a bunch of picture of the bead.. generate the sequence from that
            \item adding nucleotides generates a flash of light which is recorded.
            \item repeat nucleotides limit the length of reads - get a framechift mutation.. not good.  Still, we can get up to 1000bp. (mode 700 bp, typical throughput 700 Mb)
            \item reads per run $\approx$ 1,000,000.
            \item run time 23 hours
            \item used to be competitive when Illumina was reading length 23bp.  But now Illumina is preferred since it can read 200bp.
        \end{itemize}

    \section{Ion Torrent}
        \begin{itemize}
            \item High resolution Ph {\color{red} mirror?}
            \item essentially measuring Ph in individual wells
            \item same kind of principle as 454.
            \item 50 Mb - 1.2 Gb chips (dispoable)
            \item read length 200-400 bases
            \item 2-6 hours (fast)
            \item cheaper and easier to operate
            \item 6 hr prep, 96 off the shelf barcodes
            \item load it on to a donkey and get out in the field
            \item all of the same problems as 454 (repeat basepairs)
        \end{itemize}

    \section{PacBio: Single Molecule, Real-Tiem sequencing}
        \begin{itemize}
            \item no longer has to generate clusters (always stays focused)
            \item DNA is not immobilized.. polymerase is immobilized
            \item in principle, a 100,000 basepair sequence can be read one nucleotide at a time
            \item based on the retension of nucleotides by the polymerase
            \item fragment long double-stranded DNA fragments
            \item tack on loop-shaped adapters to make a circle.
            \item needs lots of DNA (5 micrograms of DNA) this means a single Drosophila is not enough - need 30.
        \end{itemize}

    \section{Coming up: single molecule nanopore sequencing}
        \begin{itemize}
            \item motor protien breaks double-strand
            \item threads it through a pore
            \item reads electrical current
            \item scalable modules of 2000-8000 pores
            \begin{itemize}
                \item 20 modules = human genome in 15 minutes
            \end{itemize}
            \item \$40 per Gigabase (\$3600 per 30x human genom coverage)
            \item Prototype - current error rate 15-20\%
            \item mini disposable minilop for \$900 (plugs into USB port)
        \end{itemize}

    \section{High-Capacity}
        \begin{itemize}
            \item a blessing and a curse
            \item what do we do with a 600 Gb sequence?  Or even 1 Gb?
            \item barcoding allows multiplexing
            \begin{itemize}
                \item amplification primers or adapters
                \item error-correcting
                \item one or both directions, redundant or combinatorial
                \item balancing adapters for quality calibration
                \item adapter biases, empirical testing
                \item 454 also uses gaskets to split the cells
            \end{itemize}
            \item also useful for library titration and read quantification
        \end{itemize}

\end{document}















