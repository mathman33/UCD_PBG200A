\documentclass{article}


\usepackage[margin=0.6in]{geometry}
\usepackage{amssymb, amsmath, amsfonts}
\usepackage{mathtools}
\usepackage{physics}
\usepackage{placeins}
\usepackage{enumerate}
\usepackage{cancel}
\usepackage{array}
\usepackage{color}
\newcommand{\Rl}{\mathbb{R}}
\newcommand{\cov}{\text{cov}}
\newcommand{\vari}{\text{var}}
\newcommand{\cor}{\text{cor}}
\newcommand{\expec}[1]{\mathbb{E}\qty[#1]}
\newcommand{\f}[3]{#1\ :\ #2 \rightarrow #3}
\newcommand{\prob}[1]{\mathbb{P}\qty[#1]}

\title{PBG 200A Notes}
\author{Sam Fleischer}
\date{November 4, 2016}

\begin{document}
    \maketitle

    \section{Question 4 from Ch.~2 Notes}
        \emph{Assume that humans and chimps diverged around $5.5\times10^6$ years ago, have a generation time of $\sim20$ years, that the speciation occured instantaneously in allopatry with no subsequent gene flow, and the ancestral effective population size of the human and chimp common ancestor population was $10,000$ individuals.  Nachman and Crowell sequenced $12$ pseudogeners in humans and chimps and found substitutions at $1.3\%$ of sites. \\
        \textbf{A)} What can you say about the mutation rate per site per generation at these genes, and how does it compare to other estimates of human mutation rate? \\
        \textbf{B)} All of the pseudogenes they sequenced are on the autosomes.  What would your prediction be for pseudogenes on the X and Y chromosomes, given that there are fewer rounds of replication in the female germline than in the male germline.}
        \begin{enumerate}[\bf\ \ A)\ ]
            \item Set $0.013 = 2\mu(1 - C)T$.  where $\mu$ is the mutation rate we are trying to find, $C$ is the fraction of mutations which are neutral, and $T$ is the number of generations since the common ancestor.  We can calculate $T$ by $5.5\times10^6/20 = 2.75\times10^5$.  Then solving for $\mu(1-C)$ gives us $\mu(1-C) = \dfrac{0.013}{2\times2.75\times10^5} \approx 2.\overline{36}\times 10^{-8}$.
            \item 
        \end{enumerate}

    \section{From Last Time}
        \begin{itemize}
            \item An ancestral population may split in to two separately drifting populations.
            \item Suppose both populations have current effective population size $N_e$ and the ancestral effective population size is $N_e$.  Note $F_{ST} = 1 - \dfrac{H_S}{H_T}$.  The heterozygosity of the subpopulation,
            \begin{align}
                H_S = 2N_e\times2\mu.
            \end{align}
            The heterozygosity of the total population,
            \begin{align}
                H_T = \frac{1}{2}H_S + \frac{1}{2}N_B
            \end{align}
            where $H_B$ is the heterozygosity between subpopulations.
            \begin{align}
                H_B = 2\mu\qty(T + 2N_e)
            \end{align}
            Thus
            \begin{align}
                F_{ST} = \frac{\mu T}{\mu T + 4N_e\mu} = \frac{T}{T + 4N_e}\approx \frac{T}{4N_e} \qquad \text{given }\frac{T}{N_e} \ll 1
            \end{align}
            \item $F_{ST}$ between Africans and Europeans is $F_{ST} \approx 0.095$.  Assuming $N_e = 10,000$, we can find
            \begin{align}
                T \approx 3,800 \text{ years} \approx 114,000 \text{ generations}
            \end{align}
        \end{itemize}

    \section{Island Mainland Model}
        \begin{itemize}
            \item Mean heterozygosity: hae a set of 3 loci, biallelic.  If frequency of one of them is $0.2$, then heterozygosity is $2\times0.2\times0.8$.  Do the same for $0.7$, $0.8$, take arithmetic mean, get $H_I \approx 0.35$.  For the mainland, $H_M \approx 0.47$.  We plug these in to eq.~72 in notes and get $m\approx 0.0039$.  This shows it may only takes a small number of migrations to get very low levels of heterozygosity.
            \item The rate of migration affects $F_{ST}$, as is shown in the correlation between pairwise distance between populations and $F_{ST}$.
        \end{itemize}

    \section{Population genetics of divergence between species}
        The rate of neutral mutations $\mu$ is equal to $(1 - C)\mu_T$.  The number of neural mutations at a give time point is
        \begin{align}
            2N\mu
        \end{align}
        And the probability that a particular mutation fixates is
        \begin{align}
            2N\mu \times\frac{1}{2N} = \mu
        \end{align}
        So the substitution rate is equal to the neutral mutation rate (completely independent with population size).  So the expected number of neutral substitutions in $T$ time between two species with a common ancesotr is $2T\mu$. \\

        We can use comparitive genomics to get an idea of which genes are functionally important.
\end{document}















