\documentclass{article}


\usepackage[margin=0.6in]{geometry}
\usepackage{amssymb, amsmath, amsfonts}
\usepackage{mathtools}
\usepackage{physics}
\usepackage{placeins}
\usepackage{enumerate}
\usepackage{cancel}
\usepackage{array}
\usepackage{color}
\newcommand{\Rl}{\mathbb{R}}
\newcommand{\qiq}{\ \ \ \implies\ \ \ }
\newcommand{\vari}{\text{var}}
\newcommand{\cor}{\text{cor}}
\newcommand{\expec}[1]{\mathbb{E}\qty[#1]}
\newcommand{\cov}[1]{\text{cov}\qty[#1]}
\newcommand{\f}[3]{#1\ :\ #2 \rightarrow #3}
\newcommand{\prob}[1]{\mathbb{P}\qty[#1]}

\title{PBG 200A Notes}
\author{Sam Fleischer}
\date{November 7, 2016}

\begin{document}
    \maketitle

    \section{Questions from the notes}
        \begin{itemize}
            \item Are we assuming the mean of the population is $\mu=0$?  More specifically, are we assuming nobody in the population has the gene?
            \item In figure 17, why don't we see only positive phenotypes?
            \item Question 1A:
            \begin{align*}
                \cov{\text{half-sibs}} = 0.25 = 2F_{\text{half-sibs}}V_A \qquad F_{\text{half-sibs}} = \frac{1}{8} \qiq V_A = 1 \qiq h^2 = \frac{V_A}{V} = \frac{1}{4}
            \end{align*}
            \item Question 1B: ?
            \item Who's Galton?
            \item 
        \end{itemize}

\end{document}















