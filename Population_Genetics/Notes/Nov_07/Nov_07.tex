\documentclass{article}


\usepackage[margin=0.6in]{geometry}
\usepackage{amssymb, amsmath, amsfonts}
\usepackage{mathtools}
\usepackage{physics}
\usepackage{placeins}
\usepackage{enumerate}
\usepackage{cancel}
\usepackage{array}
\usepackage{color}
\newcommand{\Rl}{\mathbb{R}}
\newcommand{\qiq}{\ \ \ \implies\ \ \ }
\newcommand{\vari}{\text{var}}
\newcommand{\cor}[1]{\text{cor}\qty[#1]}
\newcommand{\expec}[1]{\mathbb{E}\qty[#1]}
\newcommand{\cov}[1]{\text{cov}\qty[#1]}
\newcommand{\f}[3]{#1\ :\ #2 \rightarrow #3}
\newcommand{\prob}[1]{\mathbb{P}\qty[#1]}

\title{PBG 200A Notes}
\author{Sam Fleischer}
\date{November 7, 2016}

\begin{document}
    \maketitle

    \section{Questions from Ch.~3 notes pgs 42-50}
        \begin{itemize}
            \item Are we assuming the mean of the population is $\mu=0$?  More specifically, are we assuming nobody in the population has the gene?
            \item In figure 17, why don't we see only positive phenotypes?
            \item Question 1A:
            \begin{align*}
                \cov{\text{half-sibs}} = 0.25 = 2F_{\text{half-sibs}}V_A \qquad F_{\text{half-sibs}} = \frac{1}{8} \qiq V_A = 1 \qiq h^2 = \frac{V_A}{V} = \frac{1}{4}
            \end{align*}
            \item Question 1B: ?
            \item Who's Galton?
        \end{itemize}

    \section{End of the Last Leture}
        \begin{itemize}
            \item Generally $\frac{D_N}{D_S} < 1$, which is consistent with the idea that most fixed mutations are synonymous
        \end{itemize}

    \section{Inconsistencies with Neural Theory}
        \begin{itemize}
            \item Some data suggests that there are fewer nonsynonymous polymorphisms w.r.t~synonymous poymorphisms than there are nonsynonymous fixed alleles w.r.t~synonymous fixed alleles.
            \item Set $\alpha$ equal to the number of mutations due to selection, and so $1 - \alpha$ is the number of mutations due to drift.
            \begin{align*}
                (1 - \alpha)D_N = t_\text{Div}\mu_N \\
                \alpha = 1 - \frac{D_SP_N}{D_NP_S}
            \end{align*}
    \section{Incomplete Lineage Sorting}
            \item There are discrepancies between gene trees and the species tree.  This means geneticists must use multiple gene trees in order to determine phylogeny.
            \item This happens if (for example) humans and chimps don't coalesce in the human/chimp common ancestor population
            \item 
            \begin{align*}
                \prob{\text{fail to coalesce in the human/chimp common ancestor population}} = \qty(1 - \frac{1}{2N_A})^t
            \end{align*}
            where $t$ is measured in generations.  So
            \begin{align*}
                \prob{\text{incomplete lineage sorting}} = \frac{2}{3}\qty(1 - \frac{1}{2N_A})^t \approx \frac{2}{3}\exp[-\frac{t}{2N_A}]
            \end{align*}
            So given $t$, we can find $N_A$, and thus there is a $30\%$ probabiility of ILS.
        \end{itemize}

    \section{ILS vs.~Introgression}
        \begin{itemize}
            \item Look at genes which disagree with the species tree.  Consider the proportion of those which show human population 1 closer with Neandethal than with human population 2, and vice-versa.  They should be approximately the same. $N_{\text{ABBA}}$ vs.~$N_{\text{BABA}}$.
            \begin{align*}
                D = \frac{\#(\text{ABBA}) - \#(\text{BABA})}{\#(\text{ABBA}) + \#(\text{BABA})}
            \end{align*}
        \end{itemize}

    \section{Phenotypic Evolution}
        \subsection{Phenotypic Resemblance between relatives}
            \begin{itemize}
                \item Covariance, (Pearson) correlations, and slopes of linear regression
                \begin{align*}
                    \cov{X,Y} &= \frac{1}{m}\sum_{i=1}^m\qty(X_i - \overline{X})\qty(Y_i - \overline{Y}) = \overline{XY} - \overline{X}\cdot\overline{Y} = \expec{XY} - \expec{X}\expec{Y} \\
                    \cov{X,X} &= \vari{X} \\
                    \cor{X,Y} &= \frac{\cov{X,Y}}{\sqrt{\vari{X}\vari{Y}}} \\
                    \text{slope}\qty(Y\sim X) &= \frac{\cov{X,Y}}{\vari{X}}
                \end{align*}
            \end{itemize}
        \subsection{Mendel, Galton, Bateson, Fisher}
            \begin{itemize}
                \item Galton: Regression toward the mean of heights of offspring when compared to their parents heights.
                \item Bateson: Populations were evolving through large effect mutations.
                \item Fisher: We can solve the problem of continuous variation by realizing that traits are blends of many Mendelian traits.
            \end{itemize}
        \subsection{Example: BMI}
            \begin{itemize}
                \item approximately normal distribution over the population
                \item Phenotypes are always due to the interaction of geners and environments
            \end{itemize}
        \subsection{Resemblance between relatives in quantitative traits}
            \begin{itemize}
                \item A trait with $L$ loci.
                \item each segregating an allele $A_1$ at frequency $p\ell$.
                \item genotype at locus $\ell$ is $0,1,2$ with probs $p_\ell^2, 2p_\ell(1 - p_\ell),(1 - p_\ell)^2$.
                \begin{align*}
                    X_{Ai} &= \displaystyle\sum_{\ell=1}^La_lG_{\ell,i} \qiq X_{P,i} = X_{Ai} + X_{Ei}
                \end{align*}
                \item So $X_A$ is normally distributed at large enough $L$.  It's reasonable to assume $X_E$ is normally distributed.  Thus $X_P$ is normally distributed with
                \begin{align*}
                    N(\mu_A + \mu_E,V_P) \qquad V_P = V_E + V_A
                \end{align*}
                and define heritability (in the narrow sense) $h$ by
                \begin{align*}
                    h^2 \coloneqq \frac{V_A}{V_P}
                \end{align*}
            \end{itemize}

\end{document}















