\documentclass{article}


\usepackage[margin=0.6in]{geometry}
\usepackage{amssymb, amsmath, amsfonts}
\usepackage{mathtools}
\usepackage{physics}
\usepackage{placeins}
\usepackage{nicefrac}
\usepackage{enumerate}
\usepackage{cancel}
\usepackage{array}
\usepackage{color}
\newcommand{\Rl}{\mathbb{R}}
\newcommand{\qiq}{\ \ \ \implies\ \ \ }
\newcommand{\vari}[1]{\text{var}\qty[#1]}
\newcommand{\cor}[1]{\text{cor}\qty[#1]}
\newcommand{\expec}[1]{\mathbb{E}\qty[#1]}
\newcommand{\cov}[1]{\text{cov}\qty[#1]}
\newcommand{\f}[3]{#1\ :\ #2 \rightarrow #3}
\newcommand{\prob}[1]{\mathbb{P}\qty[#1]}
\newcommand{\half}{\nicefrac{1}{2}}

\title{PBG 200A Notes}
\author{Sam Fleischer}
\date{November 9, 2016}

\begin{document}
    \maketitle

    \section{Notes from Ch.~3 notes pgs 49-54}
        \begin{itemize}
            \item $\mu_S$ is the mean phenotype at reproduction (after selection)
            \item $\mu_{BS}$ is the mean phenotype before selection
            \item $\mu_{NG}$ is the mean phenotype in the next generation
            \item Set $S \coloneqq \mu_S - \mu_{BS}$ and $R \coloneqq \mu_{NG} - \mu_{BS}$.
            \item We can find $R = h^2S$ where $h$ is heritability.  This is called the Breeders equation.
            \item Using $h^2 \coloneqq \nicefrac{V_A}{V}$, we have $$R = V_A\frac{S}{V}.$$
        \end{itemize}

    \section{In class notes}
        \begin{align}
            \cov{X_1,X_2} &= \cov{(X_{1M} + X_{1P} + X_{1E})},(X_{2M} + X_{2P} + X_{2E}) \\
            &= \cov{X_{1M},X_{2M}} + \cov{X_{1M},X_{2P}} + \cov{X_{1P},X_{2M}} + \cov{X_{1PM},X_{2P}}
        \end{align}
        under some assumptions about covariances of environments.
        \begin{align}
            \cov{X_\text{mum},X_\text{child}} &= \cov{X_{1M},X_{2M}} + \cov{X_{1P},X_{2M}} \\
            &= \frac{1}{2}\cov{X_{1M},X_{1M}} + \frac{1}{2}\cov{X_{1P},X_{1P}} \\
            &= \frac{1}{2}\vari{X_{1M}} + \frac{1}{2}\vari{X_{1P}}
        \end{align}
        because the child's allele from the mother is presumably identical to one of her alleles.  So,
        \begin{align}
            \cov{X_\text{mum},X_\text{child}} &= \frac{1}{2}V_A
        \end{align}
        We can also think about $\cov{X_1,X_2}$ with $r_0,r_1,r_2$ given.
        \begin{align}
            \cov{X_1,X_2} = r_0\times0 + r_1\frac{1}{2}V_A + r_2 V_A = 2F_{12}V_A
        \end{align}
        Galton's observation is that individuals pass on their genetics, not their environment, to their offspring.

        \subsection{Predicting Offspring Phenotypes}
            $\expec{X_\text{child}} = \mu$ where $\mu$ is the mean.  However, $\expec{X_\text{child}|X_\text{mid}} = \mu + h^2\qty(X_\text{mid} - \mu)$.

            Conditions for evolution by natural selection:
            \begin{itemize}
                \item Variation must be present
                \item Survival is dependent on this phenotypic variation
                \item Variation is heritable
            \end{itemize}
            The first two are natrual selection.  The third gives rise to \emph{evolution} by natural selection.  We get rapid evolution when strong selection pressures highly heritable traits.

        \subsection{Response to Selection}
            \begin{align}
                S &= \cov{W(X),X} \\
                R &= h^2S = \frac{V_A}{V_P}S = V_A\beta \\
                \text{where } \beta &= \frac{\cov{W(X),X}}{V_P}
            \end{align}
            Lande shows
            \begin{align}
                \overline{W} &= \int W(X)P(X)\dd X \\
                R &= \frac{V_A}{\overline{W}}\frac{\dd\overline{W}}{\dd X}
            \end{align}

\end{document}















