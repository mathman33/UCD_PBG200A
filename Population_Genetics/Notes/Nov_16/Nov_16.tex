\documentclass{article}


\usepackage[margin=0.6in]{geometry}
\usepackage{amssymb, amsmath, amsfonts}
\usepackage{mathtools}
\usepackage{physics}
\usepackage{placeins}
\usepackage{nicefrac}
\usepackage{enumerate}
\usepackage{cancel}
\usepackage{array}
\usepackage{color}
\newcommand{\Rl}{\mathbb{R}}
\newcommand{\qiq}{\ \ \ \implies\ \ \ }
\newcommand{\vari}[1]{\text{var}\qty[#1]}
\newcommand{\cor}[1]{\text{cor}\qty[#1]}
\newcommand{\expec}[1]{\mathbb{E}\qty[#1]}
\newcommand{\cov}[1]{\text{cov}\qty[#1]}
\newcommand{\f}[3]{#1\ :\ #2 \rightarrow #3}
\newcommand{\prob}[1]{\mathbb{P}\qty[#1]}
\newcommand{\half}{\nicefrac{1}{2}}

\title{PBG 200A Notes}
\author{Sam Fleischer}
\date{November 16, 2016}

\begin{document}
    \maketitle

    \begin{itemize}
        \item Supposing fitness of $A_1A_1$, $A_1A_2$, and $A_2A_2$ are $1$, $1-sh$, and $1-s$, respectively, then an equlibrium frequency is
        \begin{align}
            q_\text{eq} = \frac{\mu}{sh}
        \end{align}
        \item Migration-Selection balance can preserve a near-fixation equilibrium
        \item In the simple hapliod model, migration plays the same role as mutation.
        \item If the migration rate is high enough, we get ``migration swamping'' where one allele wins for a particular set of initial conditions.
        \item interaction between selection and drift
        \begin{itemize}
            \item when selection is strong we can ignore drift
            \item \begin{align}
                p_L = P_0\times 1 + P_1\times p_L + P_2\times p_L^2 + P_3 \times p_L^3 + \dots \qiq p_L = 1 - s \qquad \text{and} \qquad p_f = s
            \end{align}
        \end{itemize}
        \item Current Status of views on molecular evolution
        \begin{itemize}
            \item the vast majority of differences between humans and chimps are due to genetic drift, not selection.
        \end{itemize}
    \end{itemize}

\end{document}















