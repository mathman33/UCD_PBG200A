\documentclass{article}


\usepackage[margin=0.6in]{geometry}
\usepackage{amssymb, amsmath, amsfonts}
\usepackage{mathtools}
\usepackage{physics}
\usepackage{placeins}
\usepackage{nicefrac}
\usepackage{enumerate}
\usepackage{cancel}
\usepackage{array}
\usepackage{color}
\newcommand{\Rl}{\mathbb{R}}
\newcommand{\qiq}{\ \ \ \implies\ \ \ }
\newcommand{\vari}[1]{\text{var}\qty[#1]}
\newcommand{\cor}[1]{\text{cor}\qty[#1]}
\newcommand{\expec}[1]{\mathbb{E}\qty[#1]}
\newcommand{\cov}[1]{\text{cov}\qty[#1]}
\newcommand{\f}[3]{#1\ :\ #2 \rightarrow #3}
\newcommand{\prob}[1]{\mathbb{P}\qty[#1]}
\newcommand{\half}{\nicefrac{1}{2}}

\title{PBG 200A Notes}
\author{Sam Fleischer}
\date{November 18, 2016}

\begin{document}
    \maketitle

    \begin{itemize}
        \item Hitch-hiking effect
        \begin{itemize}
            \item We expect to see a loss of heterozygosity as we move along the genome toward the selected allele
            \item Example in Malaria parasite - plasmodium falciparum - resistance to anti-Malaria drugs
            \item Use the width of non-heterozygous regions to give us information on the strength of selection - faster sweeping doesn't allow for much recombination.
            \item \begin{align}
                t_s = \frac{4}{s}\log(2N_e) \\
                p_{\text{NR}} = \prod_{t=1}^{t_s}\qty(1 - r(1 - x_t)) \approx \exp[-r\sum_{t=1}^{t_s}\qty(1 - x_t)]
            \end{align}
            but
            \begin{align}
                \overline{(1 - x_t)} = \frac{1}{t_s}\sum_{t=1}^{t_s}(1 - x_t) = 0.5
            \end{align}
            so
            \begin{align}
                p_{\text{NR}} &= \exp[-rt_s\overline{(1 - x_t)}] \\
                &= \exp[-\nicefrac{rt_s}{2}]
            \end{align}
            \item With probability $\pi = p_{\text{NR}}^2\times 0 + (1 - p_{\text{NR}}^2)4N\mu = \qty(1 - \exp[-rt_s])4N\mu$.
            \item Look at how far along the genome we need to see a $50\%$ return in heterozygosity.
        \end{itemize}
        \item Interaction between selected alleles and recombination
        \begin{itemize}
            \item Suppose AB and ab have high fitness, but Ab and aB have low fitness.  Then recombination is a detriment to fitness.  Recombination is general is breaking up the high fitness haplotypes for lower fitness haplotypes.  Selection and Recombination are in opposition.
        \end{itemize}
        \item Inverted genes
        \begin{itemize}
            \item Can cause ``supergenes.''
        \end{itemize}
    \end{itemize}

\end{document}















