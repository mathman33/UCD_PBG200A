\documentclass{article}


\usepackage[margin=0.6in]{geometry}
\usepackage{amssymb, amsmath, amsfonts}
\usepackage{mathtools}
\usepackage{physics}
\usepackage{enumerate}
\usepackage{cancel}
\usepackage{array}
\usepackage{color}
\newcommand{\Rl}{\mathbb{R}}
\newcommand{\prob}{\mathbb{P}}
\newcommand{\cov}{\text{cov}}
\newcommand{\vari}{\text{var}}
\newcommand{\cor}{\text{cor}}
\newcommand{\expec}{\mathbb{E}}
\newcommand{\f}[3]{#1\ :\ #2 \rightarrow #3}

\title{PBG 200A Notes}
\author{Sam Fleischer}
\date{October 28, 2016}

\begin{document}
    \maketitle

    \section{Questions from the Notes - my answers in {\color{magenta}}}
        Question 6
        \textbf{A)}
        {\color{magenta}
        \begin{align}
            \frac{\prob[G|W]\prob[W]}{\prob[G]}
        \end{align}
        where $\prob[G|W] = \prob[AA|W]\prob[bb|W]$ and $\prob[G] = \prob[G|W]\prob[W] + \prob[G|E]\prob[E]$
        }

        \textbf{B)}
        {\color{magenta}
        \begin{align}
            \text{something for now - fill in later}
        \end{align}
        }
        Question 7
        \textbf{A)}
        {\color{magenta}
        Use
        \begin{align}
            p_A = p_{AB} + p_{Ab}
            p_B = p_{AB} + p_{aB}
        \end{align}
        and eq. 21
        }

        \textbf{B)}
        {\color{magenta}
        Just average each frequency since they are combined in equal proportions.  Final answer is $D = 0.1225$.
        }

    \section{Recombination and Linkage Disequilibrium (LD)}
        A haplotype is a combination of alleles on a chromosome.  Let $r = $ the recombination fraction probability of an odd number of crossovers occur between our markers $0 < r < \frac{1}{2}$.  Define the expected frequency (expected if independent) of each of the possible haplotypes as $p_{AB}$, $p_{aB}$, etc.  Then we fine covariance (deviation from independence) as $D_{AB} = p_{AB} - p_Ap_B$, etc.  The algebra simplifies to
        \begin{align}
            D_{AB} = -D_{Ab} \qquad D_{AB} = D_{ab} \qquad D_{Ab} = D_{aB}
        \end{align}
        In a large randomly mating population,
        \begin{align}
            D_t = (1 - r)^tD_0
        \end{align}
        This is because $D_{AB} = p_{AB} - p_Ap_B$.  In the next generation (notation $D'$, $p'$), we have $p'_{AB} = (1 - r)p_{AB} + rp_Ap_B$.  Thus
        \begin{align}
            \Delta p_{AB} = p'_{AB} - p_{AB} = -rp_{AB} + rp_Ap_B = -rD
        \end{align}
        Thus
        \begin{align}
            D_{t+1} = (1 - r)D_t
        \end{align}
        If $r \ll 1$, then $1 - r \approx e^{-r}$.  Thus
        \begin{align}
            D_t \approx e^{-rt}D_0.
        \end{align}

    \section{What creates LD?}
        \begin{itemize}
            \item Mutational origin
            \item Genetic drift (and hitchhiking)
            \item epistatic selection
            \item assortive mating
            \begin{itemize}
                \item inbreeding
                \item population structure and admixture
                \item assotative mating by phenotype
            \end{itemize}
        \end{itemize}
        Mixing two populations with no LD but with different allele frequencies will cause LD.

    \section{Evolution by Genetic Drift}
        Evolution \emph{by genetic drift} is a change in allele frequency because individuals carry the allele by chance produce more or less offspring in any given generation (in sexual populations).
        \begin{itemize}
            \item Genetic drift can affect selected alleles but only if they are very weakly selected (except when they are rare)
            \item A neutral allele is an allele with no effect on fitness as compared to other alleles at the same locus
        \end{itemize}

\end{document}















